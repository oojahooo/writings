\documentclass[11pt, oneside]{article}   	% use "amsart" instead of "article" for AMSLaTeX format
\usepackage{kotex}
\title{불편함의 부재, 소통의 존재}
\author{김재호}
\date{2020.09.08}

\begin{document}
\maketitle


\begin{abstract}
    해밍의 경험처럼, 불편함은 위대한 연구의 초석이 된다. 그러나 초보 연구자는 생활 속에서 그럴듯한 불편함을 찾는 것부터 어려운 경우가 많다.
    특히 프로그래밍 언어, 소프트웨어 공학 연구는 개발 경험으로부터의 불편함이 곧 연구가 되지만, 대부분의 연구 초보에게는 그마저 부족한 경우가 많다.
    이러한 경험의 부재를 보완하여 위대한 연구를 할 가장 합리적이고 효율적인 방법은 소통일 것이다. 연구자에게는 바깥으로의 소통, 스스로와의 소통,
    연구와의 소통이 모두 중요하다.
\end{abstract}

지속적이고 구체적인 불편함은 종종 위대한 연구가 된다. 해밍의 일을 도울 프로그래머의 부재가 그의 불편함을 야기시켰고, 그것이 자동 프로그래밍이라는
아이디어의 시작이 되었던 것은 단순히 운과 우연만으로 이루어졌다고 볼 수 없을 것이다. 특히, 프로그래밍 언어와 소프트웨어 공학을 연구하는 자들에게
개발 과정의 불편함은 그대로 연구 주제가 된다. 요약 해석(abstract interpretation)과 정적 분석(static analysis)의 시작이 그러했고, 최근
무섭게 뜨는 퍼징(fuzzing) 역시 테스트케이스 생성 과정에서 겪던 불편함을 해결하기 위해 생겨났다.

그러나 안타깝게도, 연구자로서의 첫 발을 디딘 초보가 겪을 법한 개발 경험은 한정적이다. 물론 학생시절 과제를 하던 와중에도 번뜩이는 연구 주제가
나올 수 있겠지만, 대규모 프로젝트를 이끌어보거나 버그를 찾고 고쳐보는 경험은 많지 않을 것이다. 경험의 부재는 불편함의 부재를 낳는다.

이를 극복해낼 수 있는 가장 합리적이고 효율적인 방법이 소통이지 않을까 한다. 내가 경험해볼 수 없었던, 경험해보지 않았던 불편함을 간접적으로
얻어낼 좋은 자원은 주로 바깥에 있다. 지금 앉아있는 자리의 바깥일 수도, 연구실의 바깥일 수도, 더 추상적으로는 학계 바깥일 수도 있다. 그렇기에
바깥과 많은 소통을 해야 한다.

소통의 방향은 바깥만 있는 것이 아니라, 나 자신을 향할 수도 있다. 지금 내가 진정으로 중요하다고 생각하는 연구, 행동이 무엇인지 끊임없이 물어보고
되뇌야 한다. 심지어는 지금 하고 있는 연구와 소통해야 할지도 모른다. 이 연구가 진정으로 가치있을 뿐더러 충분히 가능한 시나리오인지 상상하기 위해서는,
연구의 시작과 현재, 앞으로를 떠올리고 상상할 수 있어야 가능하기 때문이다.
\end{document}